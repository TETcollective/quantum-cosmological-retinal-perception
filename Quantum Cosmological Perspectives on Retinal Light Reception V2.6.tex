\documentclass[11pt]{article}

\usepackage{amsmath}
\usepackage[utf8]{inputenc}
\usepackage[T1]{fontenc}
\usepackage[italian]{babel}
\usepackage{hyperref}
\usepackage{xcolor}
\usepackage{geometry}
\usepackage{parskip}
\usepackage{natbib}

\geometry{a4paper, margin=1in}

\hypersetup{
  colorlinks=true,
  linkcolor=blue,
  urlcolor=blue
}

\title{Quantum Cosmological Perspectives on Retinal Light Reception: \\ Mechanisms, Consciousness Implications, \\ and Therapeutic Horizons}
\author{Simon Soliman \\ tetcollective.org \\ https://tetcollective.org}
\date{Dicembre 2025}

\begin{document}

\maketitle
\thispagestyle{empty}
\newpage
\pagestyle{plain} % Solo numero pagina in basso, nessun titolo ripetuto

\begin{abstract}
La recezione della luce sulla retina non è un semplice processo biochimico, ma un sofisticato portale quantistico-cosmologico che collega la percezione sensoriale alla struttura fondamentale dell'universo. Attraverso effetti quantistici nella rodopsina (efficienza ~0.67 mediata da population splitting e coerenza vibronica) e biophotons endogeni guidati da microtubuli neuronali, la visione emerge da computazioni quantistiche orchestrate secondo la teoria Orch-OR (Penrose-Hameroff). Evidenze 2025 rafforzano coerenza quantistica e superradiance in microtubuli. Questo framework lega la percezione visiva a non-località cosciente entangled con gravità quantistica. Implicazioni mediche ampliate: rejuvenation OSK (Life Bio ER-100, trial Q1 2026 per NAION/glaucoma) preserva microtubuli retinici; optogenetics (Nanoscope MCO-010, BLA rolling 2025 per RP) bypassa fotorecettori. Tecnologiche: quantum dots photovoltaic per protesi fotoelettriche.
\end{abstract}

\section{Introduzione: Un Portale Quantistico-Cosmologico nella Percezione Visiva}

Immaginate un singolo fotone che colpisce la retina: in frazioni di femtosecondo, innesca una cascata di eventi che culmina nell'esperienza cosciente del "vedere". Questo processo non è puramente classico. L'assorbimento nella rodopsina attiva un'isomerizzazione ultrafast del retinale (11-cis → all-trans) con efficienza quantistica eccezionale (~0.67), mediata da coerenza vibronica e population splitting negli stati eccitati \citep{lorenz2022quantum}. Il segnale migra dal nervo ottico alla corteccia visiva, dove microtubuli agiscono come waveguides ottici per biophotons ultra-deboli emessi endogenamente dalla retina e dal cervello \citep{tang2014biophotons}.

Qui entra la teoria Orchestrated Objective Reduction (Orch-OR): microtubuli neuronali supportano superposizioni quantistiche di stati tubulinici, orchestrate biologicamente e collassate oggettivamente per instabilità gravitazionale (meccanismo Diósi-Penrose) \citep{hameroff2014physics}. La percezione visiva diventa un atto di entanglement con la geometria spazio-temporale cosmologica. Evidenze 2025 confermano vibrazioni coerenti THz-GHz e superradiance in reti triptofano microtubulari \citep{celardo2024superradiance}.

\section{Evidenze Quantistiche nella Visione Retinica (Aggiornamenti 2025)}

La rodopsina esibisce quantum efficiency ~0.67 grazie a splitting excited-state e coerenza vibronica, guidando la reazione attraverso conical intersections senza dissipazione eccessiva \citep{lorenz2022quantum}. Biophotons retinici/cerebrali sono canalizzati da microtubuli con coerenza Fröhlich \citep{tang2014biophotons}. Orch-OR riceve supporto da vibrazioni coerenti e superradiance triptofano \citep{celardo2024superradiance,wiest2025neurosci}.

\section{Implicazioni per la Coscienza Non-Locale}

Nei microtubuli della corteccia visiva, processi quantistici generano momenti proto-coscienti discreti, sincronizzati con oscillazioni gamma (40 Hz). Stati di flow o esperienze psichedeliche amplificano questa coerenza, dilatando il tempo soggettivo percepito e offrendo "preview" di non-località cosciente.

\section{Implicazioni Mediche Ampiate: Patologie Visive e Rigenerazione Microtubulare}

Molte patologie retiniche – glaucoma, degenerazione maculare età-correlata (AMD), neuropatia ottica ischemica anteriore non-arteritica (NAION), retinite pigmentosa (RP) – coinvolgono progressiva destabilizzazione microtubulare nelle cellule ganglionari retiniche (RGC) e fotorecettori. Lo stress ossidativo, infiammazione cronica e accumulo di aggregati proteici (es. tau iperfosforilata) interrompono trasporto assonale, plasticità sinaptica e coerenza quantistica, accelerando apoptosi e perdita visiva.

Il rejuvenation epigenetico parziale con fattori OSK (Oct4, Sox2, Klf4) rappresenta un breakthrough: in modelli NHP di NAION, ER-100 (Life Bio) – somministrato via iniezione intravitreale con doxyciclina sistemica – ripristina pattern metilazione giovanili, arricchisce pathways rigenerazione neuronale, migliora densità assoni e funzione visiva \citep{lifebio2025ardd}. Trial umani per NAION e glaucoma iniziano Q1 2026, con potenziale riduzione età biologica retinica e preservazione microtubuli per coerenza quantistica prolungata.

Complementari: senolitici (es. dasatinib+quercetina) rimuovono cellule zombie retiniche; plasmapheresis diluisce fattori pro-aging plasmatici, supportando integrità microtubulare.

\section{Implicazioni Tecnologiche/Robotiche: Protesi Visive Quantum-Inspired}

Optogenetics: MCO-010 (Nanoscope) esprime opsine multicharatteristiche in cellule bipolari, bypassando fotorecettori danneggiati. In RP avanzata, migliora acuità visiva (>0.3 LogMAR) e campi visivi; BLA rolling 2025, con durabilità 3+ anni \citep{nanoscope2025floretina}.

Quantum dots photovoltaic: PbS/AlSb QDs creano interfaces fotoelettriche flessibili, stimolando neuroni con luce ambiente senza energia esterna \citep{zenodo2025qds}.

\section{Conclusioni}

La ricezione retinica quantistica-cosmologica rivela la visione come interfaccia tra biologia e universo fondamentale. Integrando rejuvenation microtubulare con protesi quantum-inspired, apriamo orizzonti terapeutici per cecità incurabili, verso percezione cosciente eterna.

\bibliographystyle{plain}
\begin{thebibliography}{10}

\bibitem{lorenz2022quantum}
Lorenz et al., Quantum-classical simulations of rhodopsin, Nature Chemistry, 2022.

\bibitem{tang2014biophotons}
Tang \& Dai, Biophotons review, 2014.

\bibitem{hameroff2014physics}
Hameroff \& Penrose, Physics of Life Reviews, 2014.

\bibitem{celardo2024superradiance}
Celardo et al., Ultraviolet Superradiance in Microtubules, J Phys Chem, 2024.

\bibitem{lifebio2025ardd}
Life Biosciences ARDD 2025 (ER-100 trial Q1 2026 NAION/glaucoma).

\bibitem{nanoscope2025floretina}
Nanoscope Therapeutics FLORetina/AAO 2025 (MCO-010 BLA rolling RP).

\bibitem{zenodo2025qds}
Zenodo 2025, Quantum dots retinal photovoltaic prosthesis.

\bibitem{wiest2025neurosci}
Wiest, Neurosci Conscious 2025.

\bibitem{olivares2022nature}
Olivares del Campo et al., Nature Chemistry 2022.

\end{thebibliography}

\end{document}